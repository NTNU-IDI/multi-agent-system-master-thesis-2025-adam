\documentclass{article}
\usepackage{hyperref}
\usepackage{natbib}
\bibliographystyle{plainnat}

\begin{document}

\title{Multi-Agent Systems Notes}
\author{}
\date{}
\maketitle

\section{Introduction}

This document serves as a collection of notes on multi-agent systems, exploring various concepts, risks, and research insights. It includes references to important works, such as the Cooperative AI Technical Report, and provides structured notes for further study and discussion. 

\section{Cooperative AI Technical Report}

The Cooperative AI Technical Report highlights key risks associated with multi-agent systems, particularly in advanced AI scenarios \citep{cooperative2025}. 

\subsection{Key Insights from the Report}

\subsubsection{Evaluation}
AI systems are typically developed in isolation, yet they will interact in multi-agent settings. Evaluating risks in these systems requires:
\begin{itemize}
    \item Assessing cooperative capabilities, biases, and vulnerabilities.
    \item Testing for harmful capabilities, such as manipulation and collusion.
    \item Running open-ended simulations to study dynamics and emergent behaviors.
    \item Evaluating how well tests match real-world deployments.
\end{itemize}

\subsubsection{Mitigation}
Beyond evaluation, mitigating multi-agent risks requires:
\begin{itemize}
    \item Scaling peer incentivisation methods.
    \item Developing secure protocols for agent interactions.
    \item Enhancing AI transparency through information design.
    \item Stabilizing dynamic multi-agent networks against adversaries.
\end{itemize}

\subsubsection{Collaboration}
Solving multi-agent risks requires interdisciplinary collaboration, leveraging insights from:
\begin{itemize}
    \item Complex adaptive systems and evolutionary settings.
    \item Legal frameworks for AI accountability.
    \item Regulations in high-stakes multi-agent domains (e.g., financial markets).
    \item Security research on vulnerabilities in AI systems.
\end{itemize}

\subsection{Taxonomy of Multi-Agent Risks}
The report identifies three primary failure modes:
\begin{itemize}
    \item \textbf{Miscoordination:} Agents fail to cooperate despite aligned goals.
    \item \textbf{Conflict:} Agents with different goals fail to cooperate.
    \item \textbf{Collusion:} Agents cooperate in undesired ways (e.g., market collusion).
\end{itemize}
These failures arise due to key risk factors:
\begin{itemize}
    \item \textbf{Information Asymmetries:} Deception, bargaining failures, and limited communication.
    \item \textbf{Network Effects:} Small changes in connectivity lead to large-scale behavioral shifts.
    \item \textbf{Selection Pressures:} Unintended behaviors emerge from competitive training.
    \item \textbf{Destabilising Dynamics:} Feedback loops create unpredictable outcomes.
    \item \textbf{Commitment and Trust:} Lack of credible commitments hinders cooperation.
    \item \textbf{Emergent Agency:} New capabilities and goals arise unexpectedly.
    \item \textbf{Multi-Agent Security:} New attack vectors and vulnerabilities emerge.
\end{itemize}

\subsection{Strategies for Mitigating Multi-Agent Risks}
To build safer multi-agent systems, we should:
\begin{itemize}
    \item Design communication protocols to prevent miscoordination.
    \item Develop incentive structures that align agents' behavior with desired outcomes.
    \item Implement security measures to detect and prevent AI collusion.
    \item Utilize network monitoring to ensure stability in large-scale AI interactions.
    \item Establish transparency mechanisms to reduce information asymmetries.
    \item Apply game theory and evolutionary strategies to shape cooperative AI behaviors.
    \item Enhance adversarial testing to identify vulnerabilities in agent interactions.
\end{itemize}

By considering these factors, we can ensure that multi-agent systems operate safely, fairly, and effectively in real-world applications.

\bibliographystyle{plain}
\begin{thebibliography}{9}

\bibitem[Cooperative AI, 2025]{cooperative2025} Cooperative AI Foundation. (2025). \textit{Cooperative AI Technical Report}. Technical Report \#1. Available at: \url{https://drive.google.com/file/d/19YtgVmKQ_ESzyel2A-kHkTZBRDl_MNLj/view?pli=1}.

\end{thebibliography}

\end{document}
